\chapter{Abstract}

Mashups represent a new era of interactive web applications, which are quickly gaining popularity. They combine data and services to provide new functionality with emphasis on user interactivity. Mashups also harness the latest Web 2.0 tools to provide a more collaborative user experience. This thesis explores the meeting point of Web 2.0 and the Semantic Web, by focusing specifically on the emerging technologies of mashups, but also exploring semantically relevant annotations of the retrieved results, and then creating an application to illustrate the lessons learnt. We present a semantic mashup that can provide information on suitable restaurants based on a range of collated parameters. Our mashup involves integrating data from from two sources and provide categorized results based on the extracted data. We illustrate the effectiveness of the approach and the implementation with a case study and provide a basis for a more customizable functionality.

\textbf{Keywords:} python, semantic mashup, screen scraping, Web 2.0
