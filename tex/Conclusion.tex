\chapter{Conclusion}
We will conclude this thesis with some suggestions for future work and a summary of the work done.

\section{Future work}
The semantic mashup was indeed successful in extracting data from the source and the output was displayed with semantic annotations. This can be further extended to add newer features. For instance, it would be useful to add images of the restaurants along each result so that the user will have a view of its environment. It will be quite difficult, since images of these particular restaurants will be scattered across web pages, and there won’t be a single source or format available, but multiple ones. Google images API has been deprecated, but its custom search API, which also supports images can be looked into regarding this.

Since the mashup was able to scrape data off a single source, we can include another restaurant website as well using the same methodology. The other restaurant websites could support other useful features. For instance, there were some websites that can support booking a table at a restaurant and this feature can be implemented in order for the user to make reservations.

Using Web 2.0 tools, Facebook or other social networks API can be implemented. Users can log in and share recommendations with others. Reviews can be added and shared as well.

There were some delays in processing the pages due to Google maps causing load delays. Although we didn’t use Google maps API, its API can be studied and utilized in order to provide a better service of displaying locations of the restaurant. Possible integration can also specify a route via GPS from the user’s location to the restaurant. This can be looked into and it maybe well worth the effort.



\section{Conclusion}
We developed a simple and basic semantic mashup which extracts the relevant data and provides the output in a semantically enhanced way. We were also able to obtain the results needed akin to our project goals. We did have some difficulties in programming the mashup and using the necessary libraries, but it was a good learning experience in the context of this research project. Moreover, the mashup is extensible and several features specified in the previous section can be implemented to enhance the user experience.

Overall, our work demonstrates using Web 2.0 technology(such as mashups) as well as Semantic Web methods(such as annotation and metadata), and extracting the synergetic effects of these two different directions working together.


